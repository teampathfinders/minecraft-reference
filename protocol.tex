\chapter{Protocol}\label{chap:protocol}

\section{Custom data types}\label{sec:protocol-types}

Unlike the Raknet protocol, every number used in the Bedrock protocol is in little endian format.
Besides the standard integer types, the protocol uses the following custom types:
\begin{enumerate}
    \item \textit{Variable size integers} - The majority of numbers in the protocol are represented using variable size integers.
        As the name suggests, these can grow when needed.
        There are a few different types: varuint32, varint32, varuint64 and varint64.
    \item \textit{String} - Unlike Raknet's String type, Minecraft prefixes the string data with a varuint32.
    \item 
\end{enumerate}

\section{Header}\label{sec:protocol-header}

Every protocol packet is preceded by a header containing info about the packet.
This header contains the ID of the packet and the sender/target subclients.
The client or server can specify a subclient in order to send a packet to a specific player during split screen play.
Usually these IDs are set to 0 when there is no split screen.

The complete header is encoded in a single varuint32.
The first 10 bytes are the packet ID, the next 2 the sender and the last 2 are the target.

\section{Packets}\label{sec:protocol-packets}

\subsection{Request Network Settings}\label{subsec:request-network-settings}

Sent by the client to request network settings.
This is the first protocol packet sent by the client and is required in order to enable compression.

\packet{request-network-settings}{Request Network Settings (0xc1)}{
    Protocol version & uint32le & The version of the Minecraft protocol. \\
}

In case of an incompatible protocol version, a Play Status (\ref{subsec:play-status}) packet with a status of
FailedClient or FailedServer should be sent.

\subsection{Network Settings}\label{subsec:network-settings}

The network settings packet is sent in response to Request Network Settings (\ref{subsec:request-network-settings}) to tell
the client about the settings the server is using.

\packet{network-settings}{Network Settings (0x8f)}{
    Compression threshold & uint16le & Maximum allowed size of a packet before it should be compressed.
        This can be set to 0 to disable compression or to 1 to enable compression for all packets regardless of size. \\
    \hline
    Compression algorithm & uint16le & The compression algorithm to use for packet compression.
        Minecraft currently supports two algorithms: A value of 0 for Deflate and 1 for Snappy. \\
    \hline
    Throttling enabled & boolean & Enabling this allows the client to tick fewer players, improving performance. \\
    \hline
    Throttle threshold & uint8 & Minimum amount of players required before starting to throttle. \\
    \hline
    Throttle scalar & f32le & Strength of the throttling. This is probably a scale from 0.0 to 1.0? \\
}   

\subsection{Login}\label{subsec:login}

The first packet sent by the client after initiating the Raknet connection is the Login packet.
It contains most of the information that uniquely identifies each player.

\packet{login}{Login (0x01)}{
    Protocol version & uint32le & The version of the Minecraft protocol. This has been replaced by the one in Request Network Settings (\ref{subsec:request-network-settings}) \\
    \hline
    Chain data & String & The chain is a JSON-encoded array of 1 or 3 JSON web tokens. More info below. \\
    \hline
    User data & String & A self-signed JSON web token containing info such as the user's skin. \\
}

\subsubsection{Login chain}\label{subsubsec:login-chain}

The chain is a JSON-encoded array of one or three JSON web tokens.
If the user is not authenticated with Microsoft services, this chain contains one token.
It however seems that this case never occurs because Minecraft now requires an Xbox account to join servers.

In case of three tokens, the first token contains a public key in the x5u header.
This public key should be used to verify the token's signature.
It contains an identityPublicKey field which gives the public key of the second token in the chain.
As the second token is signed by Mojang in order to prevent tampering, 
this token's public key should be equal to Mojang's public key: \\

\noindent
\texttt{MHYwEAYHKoZIzj0CAQYFK4EEACIDYgAE8ELkixyLcwlZryUQcu1TvPOmI2B7vX83\\ndnWRUaXm74wFfa5f/lwQNTfrLVHa2PmenpGI6JhIMUJaWZrjmMj90NoKNFSNBuK\\dm8rYiXsfaz3K36x/1U26HpG0ZxK/V1V}
\\

\noindent
This token holds another \texttt{identityPublicKey} field that contains the public key for the third token.

The third token contains an \texttt{extraData} field, that holds the client's identity data,
such as the \texttt{XUID}, \texttt{identity} and \texttt{displayName}.
Besides the identity data, there is once again an identityPublicKey field that contains the client's actual public key.

\subsubsection{User data}\label{subsubsec:user-data}

The user data token is the largest of the four and contains mostly misc settings and the client's skin.
There is too much information to include it in this document. The token can easily be decoded by viewing it on a site like \url{https://jwt.io}.
This token is self-signed and should therefore not be seen as secure.


\section{Skin format}\label{sec:protocol-skin}