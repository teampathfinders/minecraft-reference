\chapter{Raknet}\label{chap:raknet}

Instead of TCP, which Java Edition uses, Bedrock uses the UDP protocol.
Unlike TCP, UDP does not provide any reliability or ordering guarantees:
packets may arrive in the wrong order, or not at all. 
To solve these issues, Bedrock uses the Raknet protocol. 
Raknet is a protocol that provides these guarantees on top of UDP.
The advantage of this is that, unlike with TCP, certain guarantees can be disabled if unnecessary,
reducing overhead. Sign is not known in most cases, use what is most logical.

Full documentation for this protocol is available at \url{http://jenkinssoftware.com}.
Its source is available at \url{https://github.com/facebookarchive/raknet}.

\section{Special data types}\label{sec:raknet-types}

Data types used throughout the Raknet protocol are listed below:
\begin{enumerate}
    \item \textit{String} - A string is prefixed with an unsigned short specifying its length.
    The string payload immediately follows these bytes.

    \item \textit{Magic} - The magic (or offline message data) is a sequence of bytes unique to Raknet that identifies its packets.
    It is equal to \\
    \texttt{0x00ffff00fefefefefdfdfdfd12345678}.
\end{enumerate}

\section{Packets}\label{sec:raknet-packets}

\subsection{Unconnected Ping}\label{subsec:unconnected-ping}

About every second, the client refreshes its server list.
This means that the client sends an unconnected ping to each of the servers saved in the list.
This packet has the following format:

\packet{unconnected-ping}{Unconnected Ping (0x01)}{
    Time & long & Timestamp of when this ping was sent. \\
    \hline
    Magic & 16 bytes & Raknet offline message data. \\
    \hline
    Client GUID & long & Randomised client GUID. \\
}

\textit{Client GUID} - The client GUID field is randomly generated by the client for each connection.
It is not controlled by the server and therefore, generally should not be used by the server to identify clients.
If Xbox authentication is required on the server, it is recommended to use the XUIDs instead.
These XUIDs are guaranteed to be unique and are signed by Mojang to prevent tampering.

\subsection{Unconnected Pong}\label{subsec:unconnected-pong}

An unconnected pong packet should be sent in response to the previously mentioned unconnected ping (paragraph \ref{subsec:unconnected-ping}).
This packet contains the data to be displayed in the server menu.

\packet{unconnected-pong}{Unconnected Pong (0x1c)}{
    Time & long & Timestamp of when this pong was sent. \\
    \hline
    Server GUID & long & Randomly generated server GUID. More info below. \\
    \hline
    Magic & 16 bytes & Raknet offline message data. \\
    \hline
    Metadata & string & Contains the data to be displayed in the server menu. More info below. \\ 
}

\textit{Server GUID} - It is sufficient to generate a random long at startup, as long as it stays the same between packets.
This GUID is only temporarily used by Raknet during the login sequence, it is not of any significance in the Minecraft protocol.

\textit{Metadata} - The metadata field contains all the data to be displayed by the client.
It consists of data such as the server descripton and player count.
This metadata string has a special, shown below:

\texttt{MCPE;Server Description;Protocol Version;Version Name;Player Count;Max Player Count;Server Name;Game Mode;Game Mode (numeric);Port (IPv4);Port (IPv6)}

The server name field will only be shown for LAN games.
Servers have their name set to whatever the user originally set it as while adding the server to the list.

\subsection{Open Connection Request 1}\label{subsec:open-connection-request1}

An open connection request 1 packet is sent by the client to start connecting to a server.
This is the first packet that is sent when the client joins the server.

\packet{open-connection-request1}{Open Connection Request 1 (0x05)}{
    Magic & 16 bytes & Raknet offline message data. \\
    \hline
    Protocol version & long & Currently 11. \\
    \hline
    MTU & Padding & The MTU (Maximum Transfer Unit). More info below. \\
}

\textit{Protocol version} - This is the version of Raknet that the client is using.
If this version does not match your own, send an incompatible protocol packet in return
(\ref{subsec:incompatible-protocol}). 

\textit{MTU} - The MTU is the maximum size a packet can be before any router along the network
starts dropping packets. It is the maximum size of a packet that Raknet will generate.
This MTU can be calculated by adding 28 to size of this packet. This is to account for overhead of the UDP header.

The client will continuously send copies of this packet with a decreasing padding size.
You should respond to the first of these packets that is received and send the MTU back in the reply packet.

\subsection{Incompatible Protocol}\label{subsec:incompatible-protocol}

Sent in response to an open connection request 1 (\ref{subsec:open-connection-request1}) packet if the Raknet versions do not match.
This will cause the client to disconnect, saying either the client or server is outdated.

\packet{incompatible-protocol}{Incompatible Protocol (0x19)}{
    Protocol & byte & Protocol version used by the server. \\
    \hline
    Magic & 16 bytes & Raknet offline message data \\
    \hline
    Server GUID & long & \\
}

\subsection{Open Connection Reply 1}\label{subsec:open-connection-reply1}

\packet{open-connection-reply1}{Open Connection Reply 1 (0x06)}{
    Magic & 16 bytes & \\
    \hline
    Server GUID & long & \\
    \hline
    Security enabled & boolean & Unknown what this does, it must be disabled to continue connecting. \\
    \hline
    MTU & short & MTU determined using the request packet (\ref{subsec:open-connection-request1}). \\
}